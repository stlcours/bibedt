%-*- English -*-
% Created: 12.07.05 19:14:24 by Administrator
%
% $Id: bibedt.tex,v 1.64 2006/08/13 13:04:57 stievie Exp $

% Compile this file with (may require several runs):
%   latex bibedt.tex
%   bibtex bibedt

%BibEdt
\newcommand\BibEdt{{\rm B\kern-.05em{\sc i\kern-.025em b}\kern-.09em
    {\sc E\kern-.07em d\kern-.07emt}}}
%BibTeX
\newcommand\BibTeX{{\rm B\kern-.05em{\sc i\kern-.025em b}\kern-.08em
    T\kern-.1667em\lower.7ex\hbox{E}\kern-.125emX}}
    
% Preamble
\documentclass[oneside,10pt]{article}
% The document is in english language
\usepackage[english]{babel}
%\usepackage{hyperref}
% Format URLs
\usepackage{url}
% For code environment
\usepackage{fancyvrb}
% Code environment
\DefineVerbatimEnvironment{code}{Verbatim}{fontsize=\footnotesize,frame=single,framesep=2mm,xleftmargin=5mm,xrightmargin=5mm}

% <keystroke>
\newcommand{\keystroke}[1]{$\langle\hbox{\texttt{\textsl{#1}}}\rangle$}
% Command
\newcommand{\command}[1]{\textsf{#1}}
% C++
\newcommand{\Rplus}{\protect\nolinebreak\hspace{-.07em}\protect\raisebox{.25ex}{\small\textbf{+}}}
\newcommand{\Cpp}{C\Rplus\Rplus}
% A Box
\newcommand{\hint}[2]{
	\begin{center}
		\fbox{\medskip\parbox{.95\textwidth}{
			\textbf{#1~~}\emph{#2}
		}\medskip}
	\end{center}
}
% FAQ
\newcommand{\question}[1]{\item[\textbf{Q:}] #1 \vspace{-1.5ex}}
\newcommand{\answer}[1]{\item[\textbf{A:}] #1}
% Spam stuff
\newcommand{\at}{@}
\newcommand{\dt}{.}
\newcommand{\stievieaddress}{stievie\at{}users\dt{}sourceforge\dt{}net}
\newcommand{\stieviemail}{\texttt{\stievieaddress}}
\newcommand{\nilsaddress}{hops-1\at{}users\dt{}sourceforge\dt{}net}
\newcommand{\nilsemail}{\texttt{\nilsaddress}}

% Version of BibEdt
\newcommand\VER{0.7}

% Document
\begin{document}

% The Title
\title{\BibEdt \\ Version \VER{}}
\author{Stefan Ascher\footnote{\protect\stieviemail{}}\\
  Nils Wiese\footnote{\protect\nilsemail{}}}
\maketitle

\begin{abstract}
\BibEdt{} is a simple program to edit \BibTeX{} \cite{patashnik:1988a} files and
a basic Z39.50 client for
the Win32 platform. It has a standard Windows userinterface and should be easy to
use. It allows you to add, edit and remove records from a \BibTeX{} file, as well
as editing a preamble and string macros. Additionally it features Copy and Paste
using the Windows clipboard, filter and quicksearch functionality. You can use
this program for both, source code based and dialog based editing of \BibTeX{}
files.
\end{abstract}

\tableofcontents

% We've some \verb's in footnotes
\VerbatimFootnotes

%%%%%%%%%%%%%%%%%%%%%%%%%%%%%%
% beginning of text

\section{Introduction}
\label{sec:Introduction}

A very simple program to edit \BibTeX{} files \cite{jacobson:1996}. I didn't find
something that fits my needs, and no, this program doesn't either at the moment,
but it becomes better. I found BibEdit\footnote{\url{http://www.iui.se/staff/jonasb/bibedit/}}\index{BibEdit}
to be useful, but it seems that this program is no longer maintained and the source
is not available, so I started this little program.

This program was not intended as a replacement of EndNote\footnote{\url{http://www.endnote.com/}}
or ProCite\footnote{\url{http://www.procite.com/}} (two popular commercial reference
managers), it's just a little program
to edit \BibTeX{} files. However, it's up to you to use this program as reference
manager, without using \LaTeX{}---you don't need \LaTeX{} or \BibTeX{} to run this
program. If you would like to see a feature implemented, that provides a better
intergration into your word processing program (e.g. MS Word, OpenOffice.org Writer
WordPerfect or AbiWord), feel free to make a feature request. I do not use these programs
(although I have installed them), so a small description how to achieve that should
be provided.

\section{What is \BibTeX{}?}
\label{sec:WhatIsBibtex}

\BibTeX{} is an addition to the \LaTeX{} typesetting system that makes citations
and bibliographies\index{Bibliography} real painless. It automatically picks all cited references in
a document from a database file (the \BibTeX{} file) and puts a nice bibliography
right into your final text, you don't need to sort the bibliography each time you
add a citation, it does everything for you.

If you think now: ``Buah, my Word\textsuperscript\textregistered{}\index{Word}
does not have that!'', then you should check out
Mik\TeX{}\footnote{\url{http://www.miktex.org/}} and you will have this beside much
more, e.g. a beautiful book \cite[p.~1]{knuth:1984} (or article or whatever), no longer
hassles with not functioning, incomplete beta software for an exorbitant price,
no random and/or regular crashes, no missplaced footnotes, figures, tables, no
corrupted, unrecoverable binary files and so on.

If you want to know more you could have a look at the source of this document
(included in the source distribution). It is simple and shows pretty good how to
write a basic article with \LaTeX{} and how to let \BibTeX{} append a bibliography
to the document.

\section{License}
\label{sec:License}

\BibEdt{} is subject to the GNU General Public License\index{License} Version 2 (the ``License'');
you may not use this project except in compliance with the License. You may obtain
a copy of the License at \url{http://www.gnu.org/copyleft/gpl.html}.

The \BibTeX{} lexer (\verb|lexbib|) may alternatively to the GPL also be used under the
same license as Scintilla.

\section{Installation}
\label{sec:Installation}

The easy way is to download the Setup\index{Setup} program from a SourceForge
mirror\footnote{\url{http://sourceforge.net/project/showfiles.php?group_id=142263&package_id=156662}}
which contains all required files. Install the program with a double click on the
Setup program. The Setup program additionally creates some icons to start the program.

Another way is to download\index{Download} the binary distribution of the program.
Just unzip the file \verb|bebin-ver.zip| (binary) or \verb|besrc-ver.zip|
(source)---where ``ver'' stands for the version number---to any directory and run
it from there. It will create some files in this directory the first time you start it.
When you download the ZIP package of \BibEdt{} the first time you must also download an additional
package with some libraries \BibEdt{} uses. You can download it from a SourceForge
mirror\footnote{\url{http://sourceforge.net/project/showfiles.php?group_id=142263&package_id=162662}},
it has the filename \verb|yazbin-ver.zip|. You \emph{must} download this package
and extract the files into the directory of \BibEdt{}.

\hint{Warning}{\BibEdt{} can't run without these libraries! You won't even see the
main window when you try to start it and the libraries are not present. It will
exit with an error message that YAZ.DLL and/or LIBXML2.DLL can't be found.}

\subsection{Requirements}
\label{sec:Requirements}

In order to be able to run this tiny little program you need:
\begin{itemize}
  \item A computer with an Intel x86 compatible processor.
  \item A 32~Bit MS~Windows operating system, like Windows~NT~4.0~(SP~6), Windows~2000
    and Windows~XP. Windows~9x and ME are not officially supported but it may run on
    them as well. It does currently not run on ReactOS\footnote{\url{http://www.reactos.org/}},
    I don't know why, it does not use or do something advanced. Maybe because of MFC.
\end{itemize}
There may be a need to update the following libraries on older systems, such as
Windows~95:
\begin{itemize}
  \item MS~V\Cpp~6.0 and MFC~4.2 runtime libraries.\footnote{Download
    at \url{http://support.microsoft.com/kb/q259403/}.}
  \item Common Controls at least version~5.80 (comes with Internet Explorer~5 or
    newer).\footnote{\url{http://www.microsoft.com/downloads/details.aspx?FamilyID=6f94d31a-d1e0-4658-a566-93af0d8d4a1e&displaylang=en}}
    The proper version of this library is also installed with the V\Cpp{} runtime libraries.
\end{itemize}

\subsection{Files}
\label{sec:Files}

Following files\index{Files} are included in the binary distribution:
\begin{itemize}
  \item \verb|bibedt.exe| The main program file. Start this file to execute \BibEdt{}.
  \item \verb|SciLexer.dll| Scintilla\footnote{Get the source from \url{http://www.scintilla.org/}.}
    editor control. The editor with syntaxhighlight and much more.
  \item \verb|lexbib.dll| The \BibTeX{} lexer for Scintilla.
  \item \verb|regex.dll| The regular expression library.
  \item \verb|changelog.txt| A list of changes to the program.
  \item \verb|license.txt| The GPL, the license for this program.
  \item \verb|readme.txt| A Readme file.
  \item \verb|*.xsl| Some XML styledsheet files to format the XML export, the preview
    and maybe more.
\end{itemize}
YAZ distribution contains the following files:
\begin{itemize}
  \item \verb|yaz.dll| The Z39.50 client.
  \item \verb|libxml2.dll| An XML parser used by YAZ and \BibEdt{}.
  \item \verb|libxslt.dll| The library for XSL transformation.
  \item \verb|iconv.dll| Character set conversion library used by libxml, YAZ and \BibEdt{}.
  \item \verb|zlib1.dll| Compression library used by libxml.
  \item \verb|readme.txt| A file with some instructions.
\end{itemize}
All these files are installed with the setup program.

The following files are created in the config directory (initially the directory
of the program) when you run the program the first time or this files are needed
the first time.
\begin{enumerate}
  \item \verb|config.bib| File containing informations about types and fields. Edit this
     file to fit your needs.
  \item \verb|enc_tex.txt| File containing informations about how to encode characters.
  \item \verb|enc_html.txt| Same as above but for HTML.
  \item \verb|enc_rtf.txt| For RTF.
  \item \verb|enc_xml.txt| XML.
  \item \verb|profiles.xml| to save search profiles.
  \item \verb|exporter.xml| the list of custom exporter.
  \item Maybe some more.
\end{enumerate}
Edit these files to fit your needs, the encoding files---which are mainly used by
the exporter---should be ok for all already and the profiles file is saved automatically.
If you change the configuration directory (Options dialog $\rightarrow$ page Directories),
you should move or copy these files to the new directory.

\subsection{Uninstall}
\label{sec:Uninstall}

If you've installed \BibEdt{} with a Setup program then open the Control Panel Applet
``Software'', locate \verb|BibEdt XX| (\verb|XX| is the version number)
in the list and double click on it. Alternatively you  can also use the \verb|Uninstall|
icon in the \verb|BibEdt| program group of the Start menu. This will uninstall\index{Uninstall}
\BibEdt{} with all files and registry keys.

If you didn't use the Setup program but the ZIP package then just delete the extracted
and created files and the registry\index{Registry} key:\\
\verb|HKEY_CURRENT_USER\Software\Ascher\BibEdt|.

\section{Features}
\label{sec:Features}

At the moment it does not have much features\index{Features}. Reasons may be that this project
is still at a starting point and I'm not a big \Cpp{} coder. MFC seems strange to
me since I know Borlands VCL and how easy programming could be.
\begin{itemize}
  \item Load and save \BibTeX{} files.
  \item You can use the Windows clipboard (\ref{sec:Clipboard}).
  \item It can find items with the same field values.
  \item Sort the file by key in a manner that \BibTeX{} has no problems with it
  (see~\ref{sec:CrossrefFields}). Beside that the List view may be sorted by any
  column with sort header which does not modify the file.
  \item Print and print preview.
  \item Export to various formats such as Text, RTF, HTML, XML (see \ref{sec:Export})
    and support for plug-in custom exporter (\ref{sec:AddingExporter}).
  \item Drag and Drop (\ref{sec:DragAndDrop}).
  \item Fully customizeable filter (\ref{sec:Filter}) and quicksearch (\ref{sec:Quicksearch}).
  \item Generate a \BibTeX{} file from auxiliary file (see Section~\ref{sec:GenFromAux}).
  \item Some support for the special field \verb|crossref| (see Section~\ref{sec:CrossrefFields}).
  \item Search online databases that support the Z39.50 protocoll (\ref{sec:SearchOnlineDBs}).
  \item It's a light weight program and no istallation is required. Don't be confused
    about all that DLLs, most of them are required for the online search.
\end{itemize}

So, why the hell do I take the pain and use \Cpp{} (MFC) when there is a much easier
way?! Well, that's a good point, this program would be done in Delphi in one
afternoon (I'm sleeping in the morning), but then, where would be the challenge?

\section{Usage}
\label{sec:Usage}

Just open a \BibTeX{} file (\verb|*.bib|) of your choice and edit, add, remove etc. records and fields.
Well, I think there isn't much more to say about the usage. You'll be familiar
with it soon, if you've used a Windows program at least one time.

\subsection{Views}
\label{sec:Views}

There are two ways to edit a \BibTeX{} file, a dialog and source code based way. To
edit an item visually, i.\,e. with a ``point and click'' interface, focus one in
the list and switch do the Record view. There you can edit all aspects of the item.
To edit the source switch to the Source view.

This means, this program has three so called ``Views''\index{View}: (1) the
\emph{List}\index{View!List} view, (2) the \emph{Source}\index{View!Source} view
and (3) the \emph{Record}\index{View!Record} view. To activate a certain view use
the tabs at the top of the main window. Depending on the selected view the commands
may have a different meaning. This affects especially the edit commands. For example,
when the List view is active the Copy command copies the selected item(s) in the
list, but in Source view the selected text. The same applies to the file print
commands.

Find and replace is similar, in List view you search certain field values for a
certain text, in Source view you search for a certain text in the source of the
file. You can use for both regular expressions\index{Regular Expressions}, but alas
the syntax is different. Scintilla has built in support for regular exprssions using
a certain library\footnote{Originally developed by Ozan S. Yigit at the University
of York (\url{http://www.cs.yorku.ca/~oz/}).} and the List view uses another library---the
GNU regular expression library (see Section~\ref{sec:Credits}). Well, now you
know that you must use a slightly different syntax for the two views and are not
upset that nothing is working again :-).

The List view has two panes, (1) the list itself, which shows all items, and (2)
a preview. The List view can be used to browse, view, select, remove items. Items
with missing fields have a red text color in the List view. Edit \verb|config.bib|
to change the required, optional and ignored fields of a certain type (see
Section~\ref{sec:Configuration}). The preview shows a preview of the focused item
and some fields, such as `Url', `Local-Url', `PDF',
`PS', `DOI' \cite{internationaldoifoundationidf:2005} and `crossref' (see
also~\ref{sec:CrossrefFields}), as clickable links. Clicking on a
`crossref' link will focus the referenced item, the other links will show the
linked file, URL or whatever with the systems default application. DOI\index{DOI}
addresses are automatically resolved, so you can just enter e.g. \verb|10.1000/182|
or \verb|doi:10.1000/182| as Field value and will get the URL \url{http://dx.doi.org/10.1000/182}
which will take you to the actual URL where the document is located.

The Preview is completely customizeable via XSL transformation. Edit \verb|preview.xsl|
to do so. This works similar as the exporter, a XML file is created from the focused
item and a XML stylesheet is applied. The resulting HTML file (not a physical file,
this all happens in memory only) is shown in the Preview pane. Please note that
the HTML viewer of the Preview pane is very simple and does not understand advanced
features such as CSS.

\subsection{Selection vs. Focus}
\label{sec:SelectionVsFocus}

I think it is important to make the distinction between the (one and only)
\emph{focused} item and the \emph{selected} item(s) clear. The program has always
only one selected record, except the file has no records at all, i.e. it is empty. This
is the focused\index{Focus} item (the item with the focus rectangle) in the List
view. This need not necessarily be the selected\index{Selection} one. Because more
than one items may be selected in the list, but only one can have the focus. If
you move the record pointer with one Record command only the focus is changed in
the list, not the selection. However, if you select an item with a click or with
the keyboard in the list, also the selection changes of course.

The Record view shows always the focused item, as well as the preview pane. It does not
care about the selection in the list. The same applies for the Source view, which
jumps to the focused item when you switch to it. Sayed that, it is clear that the
Record $\rightarrow$ Remove command deletes the \emph{focused} item, whereas the
Edit $\rightarrow$ Delete command deletes \emph{all selected} items in the list.
So we can say now that the Record commands operate on the focused item and the
Edit commands do something with, or change the selection.

By default only the selected items in the list are printed\index{Print} and
exported\index{Export}. If no items are selected the entire list is printed and
exported, regardless what item currently has the focus.

\subsection{Editing}
\label{sec:Editing}

Usually you edit a record in Record view. Click twice (slowly, not a doubleclick)
on the field you want to edit and a small text box will appear where you can enter
the value. In the same way you can change the name of a field.

Although editing normal fields with short values is fine in Record view, longer
values, such as for abstracts are a pain. You can use for these fields a dialog with
a multiline edit, just doubleclick (Note: clicking twice on the item shows the
inplace editor) on the field to show it. But a doubleclick on a `URL' or `Local-Url'
field opens the file or URL, in this case you can use the Edit menuitem of the
popup menu, or even the Browse menuitem if you wish to browse for a local file.
The popup menu contains some more neat commands: (\textbullet)~Crossref shows a
dialog to add a `crossref' field to an item in the file; (\textbullet)~Browse to
browse for a local file; (\textbullet)~Local URL to add a `Local-Url' field with
a filename as value; (\textbullet)~Filter
applies a filter to the item, i.e. it adds the proper field with the proper
value so that the item is visible when the file is filtered with the desired filter.

\subsection{Autocomplete}
\label{sec:Autocomplete}

By default \BibEdt{} has a feature\index{Autocomplete} turned on that you might
turn off, although it could save you much typing. If you edit a field value in
Record view it suggests\index{suggest} you values that have the same field of other
items in the file.

This could be annoying especially for fields that can have a very long value,
such as the abstract field. Turn it off with the Options dialog $\rightarrow$ page
General $\rightarrow$ uncheck Autocomplete field values. You can temporarily disable
it with the key combination \keystroke{Shift+Escape} in the inplace editor.

\subsection{Clipboard}
\label{sec:Clipboard}

\BibEdt{} uses its own clipboard\index{Clipboard} format\index{Clipboard!Format},
but if you have the source of one or more \BibTeX{} records as text in the clipboard
you can directly paste it into \BibEdt{} with the Edit $\rightarrow$ Paste\index{Paste}
command as well. If you paste a record, \BibEdt{} will check if \BibEdt{}'s own format
is available and use this if possible, if not it tries the plain text format. Pasting
records in plain text format may fail if the program can't parse it.

If you copy an item it puts it as (\ref{ei:cfbibedt}) \BibEdt{}'s clipboard
format, (\ref{ei:cfplaintxt}) as plain text (so you can paste it into any text
editor), as (\ref{ei:cfrtf}) RTF\index{Clipboard!RTF} and (\ref{ei:cfhtml}) HTML
(to paste it directly into a word processing program) to the clipboard. Usually
the target program decides what format it likes most, Notepad will prefer plain
text and a word processor most likely RTF or HTML. If your word processor supports
HTML, I would prefer it over RTF.
\begin{enumerate}
  \item\label{ei:cfbibedt} \BibEdt{} clipboard format is a plain text format of
    the source of the item. This format is only used by \BibEdt{} internally. If
    you paste an item, and this format is available, it uses this, if this format
    is not available it tries the plain text format.
  \item\label{ei:cfplaintxt} Plain text of the source of the item. You can use this
    to paste the item(s) as \BibTeX{} source.
  \item\label{ei:cfrtf} Formated RTF as defined in \verb|config.bib|. In the future
    this also will be done with XSL transformation.
  \item\label{ei:cfhtml} Formated HTML as defined in \verb|copyhtml.xsl|.
\end{enumerate}

\subsection{Drag and Drop}
\label{sec:DragAndDrop}

This program supports the following Drag and Drop operations:
\begin{enumerate}
  \item You can drag a file from Explorer and drop it into the List view. This will
    load the file.
  \item You can drag a file from Explorer and drop it into the field list in
    Record view. This will add a field with name `Local-Url'\index{Local-Url} and the filename as
    value. The program converts backslashes to forward slashes\footnote{Windows
    can handle forward slashes as directory separators quite well.} and add the \verb|file:///|
    protocoll at the beginning: \verb|file:///D:/foo/bar.pdf|. Doubleclick on such a field
    in the list, or on a `Url'\index{Url} field will open the file or URL. Also the preview
    pane shows these fields as clickable links. More filenames are concatenated
    with a \verb*|; |.
  \item The source editor supports usual drag and drop editing. Select some text
    and move it to another location.
\end{enumerate}

\subsection{Filter}
\label{sec:Filter}

First of all, filter\index{Filter} are great! They provide a way to filter the file. Every file may have its
own set of filter. They are saved inside a \verb|@comment| tag, because these tags are
ignored by \BibTeX{}. If you filter the file, it basically searches a certain field
for a certain token. If it finds the token it shows the item, all other items are
hidden. The Source view shows always the entire file.

You can add, edit and remove filter for a file with the Filter dialog\index{Filter!Dialog}.
Use the Filter menuitem in the Tools menu to show this dialog. Click the Add button,
enter a name for the filter, and which field should contain what string. As all
field names, the filter name must also not contain spaces and special characters. To
apply a filter use the Filter ComboBox beside the toolbar. Using \command{[No Filter]}
shows all items.

A bibitem may belong to more categories\index{Cathegory}, let's assume you have a filter like:
\begin{code}
@comment{__filter__,
  Psychologie={Cathegory=Psychologie},
  IT={Cathegory=IT},
  Statistik={Cathegory=Statistik},
  Tests={Cathegory=Tests},
  Biologie={Cathegory=Biologie},
  Soziologie={Cathegory=Soziologie}
}
\end{code}
then a bibitem like:
\begin{code}
@article{baron+kenny:1986,
  author={R. M. Baron and D. A. Kenny},
  title={The Moderator-Mediator Variable Destinction in Social
    Psychological Research: Conceptual, strategic, and statistical
    considerations},
  journal={Journal of Personality and Social Psychology},
  volume={51},
  year={1986},
  pages={1173--1182},
  Cathegory={Psychologie, Statistik}
}
\end{code}
whould be shown when you apply the filter ``Psychologie'' \emph{and} ``Statistik'',
because the Cathegory field contains both words.

The filter is not limited to a field. The program consideres the type of an item
as virtual\index{Field!virtual} field named \verb|Type|\index{Type} and the key as
field with name \verb|Key|\index{Key}, so using a filter like:
\begin{code}
@comment{__filter__,
  Articles={Type=article}
}
\end{code}
would show all articles in the file (the `@' is not part of the type). In the
Filter dialog you can enter \verb|Type| as Field and \verb|article| in Contains.

If the file is filtered and you add a new record to the file, it automatically
applies the filter to the new item. Appending another file with the Import command
does \emph{not} import the filter defined in the imported file. You could use the Source
view to merge the filter from the two files without too much clicks.

You can define a set of default filter in \verb|config.bib|. The syntax is the same
as for ordinary \BibTeX{} files, it looks like an usual bibitem but is placed
in a \verb|@comment| with name \verb|__filter__|. All default filter are appended
to filter defined in the loaded file.

\subsection{Quick search}
\label{sec:Quicksearch}

Quick search\index{Quicksearch} is similar to filter, but you don't define it before
and choose then a filter, you enter the text to search for in an edit box and the
programs shows only matching items. Invisible items, e.g. because you've applied
a filter, are not searched and not shown even if one would match. Only items that
are visible in the list are searched.

Define the default fields\index{Quicksearch!Fields} to be searched in the Options
dialog $\rightarrow$ page General. All fields
must be separated with a semicolon (\verb|;|) and at the end there must also be such a
thing. Example: \verb|key;author;title;| would search the Key, the Author and Title field.
By default it uses this defined fields to search, but you can enter in the ComboBox
directly what fields to search. More fields must be separated by a semicolon as
in the Options dialog. To search all fields and the Key, but not the Type, choose
\command{[All Fields]} in the ComboBox.

Similar to filter, you are not limited to fields of items, you can also search
for a type or a key of items, just enter \verb|Type| or \verb|Key| in the Fields ComboBox.
For example, if you want to see all \verb|inbook|s enter in Fields \verb|Type| and
in the search textfield \verb|inbook|.

If you wish to search with regular expressions check the Regular expression menuitem
of the Search menu.

\subsection{Generate a \BibTeX{} file from an auxiliary file}
\label{sec:GenFromAux}

The program can scan auxiliary\index{Auxiliary} files emited by \LaTeX{}---such files contains all
keys cited in a document---and generate from it a subset of a main database.

To make it short, choose the Tools $\rightarrow$ Generate from AUX command, enter in the
top textbox the toplevel auxiliary file, enter in the second textbox the target file
and click Generate. This will create the target file from the currently open file
only with the items cited in the document and with items referenced by cited items.
If \command{Only required and optional fields} is checked in the dialog it removes
all ignored and unknown fields from the resulting file.

It is important that you enter the \emph{toplevel} auxiliary file, because only
this file contains the citations. This file has the same filename as the main \LaTeX{}
file, but with the the extension \verb|.aux|.

\subsection{\texttt{crossref} fields}
\label{sec:CrossrefFields}

This program handles the \verb|crossref|\index{Crossref} field slightly different, see
\cite[p.~2]{patashnik:1988a} what's all about with this field---very useful when
you have much \verb|inbook|s or similar.
\begin{enumerate}
  \item You can easily add a reference to another item in the open file with the
  popup menu of the field list (Record view) $\rightarrow$ Crossref. It will show
  a dialog with all keys in the file. Selecting a key will add the field with the
  selected key as value. If there is already such a field it replaces the old key
  with the new, because only one can be referenced.
  \item Similar to the URL and Local-Url fields it shows a clickable link in the
  preview pane. Clicking on it will focus the referenced item.
  \item Missing fields of items are filled with the fields of the referenced item
  when you export to XML or descendeds.
\end{enumerate}

If you sort the file with the Record $\rightarrow$ Sort command all items that are
cross-referenced by another item are put at the bottom of the file. It may be required
that you use the Sort command twice, so it can identify all items that are referenced.
This means the file isn't entirely sorted, but this is required by \BibTeX{} to
work correctly. However, if you want the file sorted in the list you could use the
sort header of the List view, this doesn't touch the file.

\subsection{Serarch online databases}
\label{sec:SearchOnlineDBs}

You can search online databases which supports the Z39.50~\cite{niso:2003} protocoll. Go to Tools
$\rightarrow$ Online search, enter some informations and the query string and click
Search to start the query. The list at the bottom lists the results. Enter in the
Show from~\dots{} and to~\dots{} textfields the record range to display. The first
record starts with number 1. Enter \verb|0|
as upper bound to show all records. The Next button will show the next records.
The Import button imports the selected items in the list into the open document.

You can save your access data---i.e. Hostname, Portnumber, Proxyhost etc.---in a file,
so you don't need enter everytime the same you do a query. For this you need to
create a new profile with the New button at the top of the Dialog, then enter a
name for the profile. From now on the program fills in the data when you select
a profile in the dropdown list the next time. Everything is saved, except the
query string, also the password if you wish, but it is saved \emph{plain} to the
file. You can choose the
file in which the profiles are saved with the Options dialog $\rightarrow$ Files
$\rightarrow$ Search profiles file.
\hint{Warning}{The password is saved plain to the profiles file and everyone who
has access to your computer can read it.}

For the moment only response in USMARC (MARC~21) \cite{ndmsoffice:2004} format
is supported, more formats may follow in the future.

Index Data has a huge list\footnote{\url{http://www.indexdata.dk/targettest/}} of
known server supporting the Z39.50 protocoll with all informations you need to know
to access these server with \BibEdt{}.

\subsubsection{Query types}
\label{sec:QueryTypes}

There are two different query types. These are ``PQF'' (Prefix Query Format) and
``CQL'' (Common Query Language). Please consult the YAZ Users Guide for a description
of the grammar located at \url{http://indexdata.dk/yaz/doc/} or as PDF at
\url{http://indexdata.dk/yaz/doc/yaz.pdf}. The CQL format is
straight forward, there you can simply enter for example \verb|author=knuth|.

\subsection{Export}
\label{sec:Export}

With \BibEdt{} you can export a file or a subset of a file as many different formats.
To export the file use the Export menuitem of the File menu. Choose in the apperaring
file save dialog the filename and the export format. You need not enter a file extension,
the exporter does that for you.
\hint{Warning}{If a target file already exists, it'll overwrite it without asking
you anything. The file save dialog doesn't warn you either.}

If there is a selection in the List view it exports only the selected items. If
there is no selection it exports the entire file. This way you can create a subset
of the open \BibTeX{} file, just select the items you wish to export and choose
as exporter \command{BibTeX}. Another way to create a subset of the open file---maybe if
you wish to ship a \BibTeX{} file with only the items you cited in a certain
document---is to let \BibEdt{} scan your AUX file and create a \BibTeX{} file with
only the cited items in the document, see Section~\ref{sec:GenFromAux} for details.

It has some built in exporter, such as plain text or XML\index{XML} \cite{w2c-xml:2004},
and you can have custom exporter, which basically transforms the XML export with
a XSL file to an other format. This has some significant advantages over built in exporter, because
you're completely free how the final result will look, you just need to modify
the respective XML stylesheet (XSL)\index{XSL} file \cite{w3c-xsl:2001}. You use some kind of language
to extract the informations you want to show from the XML file. The extracted fields
can be formatted with the usual HTML tags. Although this is not the place to
teach you XSL and XPath \cite{w3c-xpath:1999}, here is a simple example of the
body of an XSL file (for a ``real world'' file see \verb|plain.xsl|, which is used
for the HTML export):
\begin{code}
<!-- Iterate through all bibitem tags within the bibliography tag. -->
<xsl:for-each select="bibliography/bibitem">
  <!-- Read the type field into variable item-type. -->
  <xsl:variable name="item-type" select="type"/>
  <!-- Depending on the type of the item use a different format -->
  <xsl:choose>
    <xsl:when test="$item-type = 'article'">
      <!-- It's an article -->
      <xsl:value-of select="author"/>  <!-- Print the author -->
      <span style="font-style:italic;">
        <xsl:value-of select="title"/> <!-- Print the title italic -->
      </span>
      In: <xsl:value-of select="journal"/> <!-- Print the journal -->
      <!-- and so on ... -->
    </xsl:when>
    <xsl:when test="$item-type = 'book'">
      <!-- It's a book -->
      <!-- Do something with the fields -->
    </xsl:when>
    <--! Handle other types! -->
    <xsl:otherwise>
      <!-- It's something else -->
      Unknown type!
    </xsl:otherwisw>
  </xsl:choose>
</xsl:for-each>
\end{code}
If you edit a XSL file while \BibEdt{} is running, you should restart the program,
because it loads the XSL file when it is first needed and keeps it then in memory
for performance reasons.

This program comes with a default XSL file (\verb|plain.xsl|), which tries to
mimic the plain \BibTeX{} style and emits regular HTML. There are other XSL files
included in the distribution, to produce other formats than HTML, like RDF, RTF,
plain text.
If you've created your own XSL file you can tell \BibEdt{} what file to link to
the XML export with the Options dialog $\rightarrow$ Files $\rightarrow$ XML stylesheets file. If you
want to provide your XSL style file to other users submit a
patch\footnote{\url{http://sourceforge.net/tracker/?atid=753204&group_id=142263&func=browse}}
and attach the file. I'll include it then in the binary distribution.

\hint{Note}{Some browser may habe problems interpreting and rendering the exported
file, but Firefox does it right.}

You can translate the XML file into plain HTML with some external tools as well,
such as Saxon\footnote{\url{http://saxon.sourceforge.net/}}\index{Saxon},
or xsltproc\footnote{\url{http://xmlsoft.org/XSLT/} download
\verb|libsxlt-version+.win32.zip| (\verb|version| denotes the version number)
from \url{http://xmlsoft.org/sources/win32/}.}. If you call Saxon with the arguments:
\begin{verbatim}
saxon.exe -a -o "result.html" "infile.xml"
\end{verbatim}
it'll create \verb|result.html| from the source XML file \verb|infile.xml|. Meaning
of the commandline options:
\begin{description}
  \item[-a] Use the linked stylesheets file.
  \item[-o] Write result to file rather than to the standard output (StdOut).
\end{description}
Saxon is also useful for debugging XSL files because the error messages are more
meaningful than that of Firefox.

\subsubsection{Adding exporter}
\label{sec:AddingExporter}

To add a custom exporter use the Options dialog $\rightarrow$ page Exporter. There
you must choose which XSL file should be used for the transformation, the character
encoding, file extension and file filter. The dot belongs to the extension, so
\verb|.txt| should be used for text files, not \verb|txt|. If you omit the dot
the program adds it for you.

The built in exporter can't be removed or edited. For a HTML exporter use the
\verb|plain.xsl| file, for a RDF exporter the \verb|xml2rdf.xsl| file etc. There
are other XSL files distributed with the program, but not all are exporter, and
some are for internal use, e.g. \verb|preview.xsl| for the preview.

\subsection{DDE}
\label{sec:DDE}

To be able to control a program with DDE\index{DDE}\footnote{A technique for
sharing memory and interprocess communication.} you need a Server name and (optional) a
Topic, use the informations bellow:
\begin{description}
  \item[Server]\index{DDE!Server} \verb|bibedt|
  \item[Topic]\index{DDE!Topic} \verb|System|
\end{description}

In the DDE message descriptions, the square bracket characters \verb|[| and \verb|]|
in DDE messages are significant, and must be included as part of the message.
The following DDE commands\index{DDE!Commands} are supported:
\begin{description}
  \item[open] \verb|[open("filename")]|: Open the file with \verb|filename|.
  \item[print] \verb|[print("filename")]|: Print file with \verb|filename| to default
    printer.
  \item[printto] \verb|[printo("filename","printername","driver","port")]|: Print file with \verb|filename|
    to printer with \verb|printername| and \verb|driver| on \verb|port|.
  \item[goto] \verb|[goto("filename","itemkey")]|: Open file with \verb|filename| and
    focus item with key \verb|itemkey|.
  \item[focus] \verb|[focus("itemkey")]|: Focus the item with \verb|itemkey| in the
    open file.
  \item[find] \verb|[find("string")]|: Quicksearches \verb|string| in selected fields.
  \item[filter] \verb|[filter("name")]|: Sets filter with \verb|name|.
\end{description}

\subsection{Command Line}
\label{sec:CommandLine}

Alternatively to DDE you could also use the command line\index{Commandline}. The syntax is as follows:
\begin{verbatim}
bibedt.exe [[/<action>] <filename> [<driver> <port>]]
\end{verbatim}
Where \verb|action| can be:
\begin{description}
  \item{\verb|p|:} Print the given file and exit.
  \item{\verb|pt|:} Print given file to a certain printer with certain driver on
    certain port.
  \item{\verb|dde|:} Execute a DDE command.
\end{description}
If no action is supplied the program will just open the file. If an action is supplied
also a filename must be supplied. If neither nor is supplied the program starts
with an empty file. The following example opens the file, prints it to the default
printer and exits:
\begin{verbatim}
bibedt.exe /p "myfile.bib"
\end{verbatim}
As ever, filenames containing whitespace (space or tab) must be surrounded by double
quotes.

\section{Configuration}
\label{sec:Configuration}

\subsection{\texttt{config.bib}}
\label{sec:configbib}

There is not much to configure, except the \verb|config.bib| file is maybe worth
some words. This file defines the types of bibliography items, how to format them
for printing, exporting and the preview. A detailed explamation of the file can
be found in the file itself, it is pretty good commented, in my opinion. You should
\emph{not} use \BibEdt{} to edit this file, because it's not a regular \BibTeX{}
file, use an ordinary texteditor instead.

The program loads the file at startup and keeps it then in memory. So, if you have
modified the file, modifications will not take affect until you start the program
the next time. If you want the original file back, then just delete it. The program
automatically creates the default file, if it can't find it.

\subsection{Insert selected key into an Editor}
\label{sec:InsertKey}

Another feature requires some description, namely how to insert the selected key
into your \LaTeX{} editor with one click. First you must tell \BibEdt{} what's your
default editor: Open the Options dialog, go to the Editor page and enter the window
class name of the top-level and the child window. You can find the class names of
windows with programs like GetWnd\footnote{\url{http://web.utanet.at/ascherst/files/getwnd.zip}
(200~KB)}.
In this program the first item in the list is the top-level window and the last
the child window you should enter, when you move the mouse over the editor area
of your editor. Then you can just use the Insert Key command
of the Record menu. Your program (more precicely the child window) must handle the
\verb|WM_PASTE| window message, because it copies the selected key to the clipboard
and sends then the \verb|WM_PASTE| message to the window, if the program has found
one. Of course, the program must run to be able to find the window. As far as I can see WinEdt
does not handle this message, so this does not work for you if you use WinEdt.
It doesn't work with TeXnicCenter either, because MFC uses window class names in
a strange way, but it works with the syn Text Editor\footnote{\url{http://web.utanet.at/ascherst/syn/}.
I noticed the program freezes on XP frequently when you save a file, but I've no
time to fix that.}.
Enter \verb|TSynMainForm| as top-level window and \verb|TSynEdit| as child window
if you use syn. Sorry, but I don't know any other method to insert text in an editor
which is unknown at designtime.

\subsection{Configuration directory}
\label{sec:ConfigurationDirectory}

If more users on one copmuter use this program and wish to have different configurations
you should change the configuration directory from the program directory to somewhere
in the users application data directory\footnote{On Windows~NT usually
\verb|C:\WINNT\Profiles\<username>\Application Data|, on Windows~2000 and later
\verb|C:\Documents and Settings\<username>\Application Data|.} (Options $\rightarrow$ Directories $\rightarrow$
Configuration directory). Before that you should create this directory
and copy the configuration files into it. Changing this directory may require a
restart of the program, because some of the files are loaded at program start, and
kept then in memory.

\section{Localization}
\label{sec:Localization}

\BibEdt{} can be localized with resource DLLs. I do not provide such DLLs, so if
you want \BibEdt{} in your language you've to make the DLL for your language with
the translations of the original resources---Dialogs, Menus, Toolbars, Strings etc. See the MSDN library
how, especially the Articles Q198846 \cite{microsoft:2004}, Q147149 \cite{microsoft:2004a} and
TN057 \cite{microsoft:2004b}.
The DLL must have exactly the same version number as \BibEdt{} and in the Comment
field of the version resource must be the name of the language (e.g. German).

To use your resource DLL make a subdirectory with the name \verb|lang| and put it
in there. Then open the Options dialog, go to the Language tab and select your
language. You may need to restart \BibEdt{}.

If you have problems with your resource DLL you may face strange effects and it's
possible that you can't change the language back with the Options dialog, maybe
because \BibEdt{} doesn't even start. In such a case start RegEdit open the key
\verb|HKEY_CURRENT_USER\Software\Ascher\BibEdt\Settings| and delete the value of
\verb|Language|. Then you'll have the built in english language again.

\section{Warning}
\label{sec:Warning}

Use this program with care and always make a backup copy of the file before you
save it. The parser may still contain bugs and/or does not support some feature
\BibTeX{} supports. The program creates by default a backup copy before you save
a file, but you can turn it off. The backup file has the same filename but with
the extension \verb|.~bib|.

You've been warned, the rest it up to you. However, I use this program without
major problems, e.g. it didn't destroy my \BibTeX{} files so far.

If you've problems with this program use a tracker at SF\footnote{\url{http://sourceforge.net/projects/bibedt/}}
or the forum, or use another program.

\section{Limits and Problems}
\label{sec:LimitsAndProblems}

There are the following limits:
\begin{itemize}
  \item Only curly brackets\index{Brackets} can be used, round brackets will confuse the program:
\begin{code}
@article(key,
  foo=(bar)
)
\end{code}
    does not work, it should be:
\begin{code}
@article{key,
  foo={bar}
}
\end{code}
  \item All field values must be enclosed in curly brackets (\verb|{}|). The following
  code is wrong and will not work, although it might work for \BibTeX{}\footnote{Sometimes
  I think \BibTeX{} must use some bayesian \cite{bayes:1763} algorithm to parse
  such files. Indeed, it seems hard to me to write a robust parser for \BibTeX{}
  files!}:
\begin{code}
@article{key,
  foo="bar"
}
\end{code}
  \item Forename must appear before the surename, so it should be:
\begin{code}
@article{key,
  ...
  author={Forename Surename},
  ...
}
\end{code}
  and not:
\begin{code}
@article{key,
  ...
  author={Surename, Forename},
  ...
}
\end{code}
  Multiple names must be combined with \verb|and|. Additions to the name must be
  grouped, e.g. \verb|Ludwig {van Beethoven}|, \verb|Pierre {La Mure}|,
  \verb|Artemio {Ramirez Jr.}|. Everything that does not have fore- and surenames,
  such as company names, organisations etc., must be enclosed in braces, e.g.
  \verb|{Intel Corporation}|. Other formats will confuse the key generation algorithm,
    but does not have any other effects.
  \item Comments\index{Comment}, marked with \% are not written to the file when you save it, you
    could use \verb|@comment{...}| instead. In \BibTeX{} everything not starting
    with a `@'\index{@} is a comment. This program will skip everything not starting with `@'. This
    means everything that does not start with a `@' is stripped from the file when
    you save it. It would be hard to fix this issue.
  \item The \verb|key| and \verb|type| fields are special fields, I know this conflicts
    with \BibTeX{}, I'm working on that\dots{}
  \item It should maintain the order\index{Order} of the items. However, if you
    use the strings\index{Strings} or preamble dialog strings and preambles\index{Preamble}
    are always put to the top of the file.
    This may cause problems when you use string macors in a preamble. The order
    of real bibitems are not thouched, unless you sort the file. If you sort the
    file \verb|@string|s are put before any preambles, so you can use string marcos in
    preambles.
  \item \TeX\ escape sequences must be enclosed in curly brackets, e.g. \verb|{\"a}|,
    \verb|\"a| has no negative effect, but won't be translated into an `\"a'.
  \item In my humble opinion, there shouldn't be a comma after the last field:
\begin{code}
@article{key,
  ...
  foo={bar},
  lastfield={some value},
}
\end{code}
    However, if there is such a thing, \BibEdt{} just removes it.
  \item The program will produce unpredictable results when a file is opened \BibTeX{}
    reports errors.
\end{itemize}
Ok, this is quite a lot, but I'm working at least on some of these problems.

\section{FAQ}
\label{sec:Faq}

Some more or less frequently asked questions, or questions that might be of interest.

\begin{itemize}
  \question{What does this stupid program do?}
  \answer{You can use this stupid program to edit \BibTeX{} files. If you have
    nothing to do with \BibTeX{} you should either change this quickly or this
    program is most likely useless for you.}

  \question{This program crashes sometimes, was this intended?}
  \answer{No.}
  
  \question{And how about freezing?}
  \answer{No.}

  \question{Other bugs?}
  \answer{No.}

  \question{The ugly interface?}
  \answer{Yes.}

  \question{Are there better programs that do the same?}
  \answer{Check out JabRef\footnote{\url{http://jabref.sf.net/}} or BibDesk\footnote{\url{http://bibdesk.sf.net/}}
     if you use Mac.}

  \question{Do you add feature XXX?}
  \answer{Open a feature\index{Feature} request at the SourceForge project page\footnote{\url{http://sf.net/projects/bibedt/}}.}

  \question{Do you fix bug XXX?}
  \answer{Make a bug\index{Bug} report at the project page.}
  
  \question{Isn't is useless to make this program localizeable but do not provide
    language DLLs? I mean, if I would be a programmer and would have the tools,
    I would write such a program for my own and it would be much better.}
  \answer{Do what you want.}

  \question{Copying a lot of items in the list takes a long time, why is that the case?}
  \answer{It iterates through the list and copies\index{Copy} the selected items as four different
     formats. That are (1) plain text, (2) \BibEdt{}'s own clipboard format,
     (3) formated RTF and (4) formated HTML. RTF and HTML are useful because so
     you can paste it directly into your word processing program.}

  \question{I run this program on XP\index{XP}, but the tabs\index{Tab} of the mainwindow doesn't seem to have
     the usual style as tabs on XP have. What's wrong?}
  \answer{This is not a normal windows tab control, it is not a tab control at all, these
     are just some lines in different colors drawn on the main window. So nothing
     is wrong, this will look on all windows versions exactly the same.}

  \question{Some things only work poorly, or don't work at all. Are you sure that
    you want to release this program.}
  \answer{Well, things may become better when there is something different than a `0'
    as major version number.}
    
  \question{When will the first non-Beta version be available?}
  \answer{Uhm\dots{}}
  
  \question{``Beta'' means usually a feature freeze. Does this mean you don't add
    new features in the near future?}
  \answer{Who sayed that? IMHO, ``Beta'' means it could work, but it need not.
    \BibTeX{} is beta since more than 20~years.}

  \question{May I contribute to this \emph{excellent} project?}
  \answer{Sure, mail me what you want to do.}
  
  \question{How do I download the sources via CVS?}
  \answer{SourceForge has a good explanation of CVS\index{CVS}
    itself\footnote{\url{http://sourceforge.net/docman/display_doc.php?docid=14033&group_id=1}}
    and provides all informations you must know for a certain
    project\footnote{\url{http://sourceforge.net/cvs/?group_id=142263}}.
    If you ask me what's the easiest way to get the sources then download
    TortoiseCVS\footnote{\url{http://www.tortoisecvs.org/}}, install it and do a
    ``Checkout'' of the \emph{Module} \texttt{bibedt}, the rest shoudn't be a problem.
    Note: We use currently CVS and not SVN.}
  
  \question{May I donate money to this awesome program?}
  \answer{Sure, just got to \url{http://sourceforge.net/donate/index.php?group_id=142263}
    or \url{http://www.doctorswithoutborders.org/} or
    \url{http://www.amnesty.org/} or such. If you donate to \BibEdt{} 10\,\% of your
    donation will receive SourceForge for their excellent service.}

  \question{Is there really a need for this program?}
  \answer{Is there a need for \emph{any} program???}
\end{itemize}

\section{Support}
\label{sec:Support}

Support\index{Support} is provided via public forums, so that other users can benefit
from you questions and reports. In special, rare cases, e.g. if you don't like the
publicity of such a forum, you may mail me directly. I encurage everybody to encrypt
their mails, not at least to reduce the risk that I delete your mail by accident---Spam
became a big problem. My public GnuPG\footnote{\url{http://www.gnupg.org/}}
key\footnote{The fingerprint is \verb|10D4 A724 7B54 253A B8D7 2E54 1F5F 0FFD 2195 2A98|.}
can be found at \url{http://bibedt.sourceforge.net/stievie.asc}.

If you've found a bug then make a bug report, if you would like to see a feature
in a future version make a feature request. In either case, please provide a \emph{detailed}
description of what's wrong or what would be great, and maybe ideas how to accomplish
that. In case of a bug, it would also be interested to know what you did before
you noticed the problem.

\section{To do}
\label{sec:ToDo}

\begin{itemize}
  \item Characterconversion does not work very good. All conversion should be done
    by iconv, it is possible to add custom charactersets (\BibTeX{}, RTF, HTML
    $\leftrightarrow$ current systems characterset), but at the moment iconv does
    not like my attempts doing that.
  \item Print and Print Preview for the List and Source view is buggy.
  \item Resonse from a Z39.50 server can vary very much. The program should be able
    to handle most of them. The response should arrive in XML and then should be
    transformed with XSL into a uniformed and simplified XML which \BibEdt{} can understand. This
    way only an XSL file will be added when a server response cannot be handled.
  \item It should be possible to hide the Preview pane. Unfortunately this does
    not do MFC.
\end{itemize}
For the far future:
\begin{itemize}
  \item Undo/Redo for the list view. Ouch, this sounds hard.
  \item The \BibTeX{} parser should become more robust. It should be able to handle
    all files \BibTeX{} accepts.
\end{itemize}

\section{Credits}
\label{sec:Credits}

Code not written by me:
\begin{itemize}
  \item Neil Hodgson for Scintilla, the source editor control.
  \item Horst Br\"uckner for the MFC interface to Scintilla, found at CodeProject\footnote{\url{http://www.codeproject.com/}}.
  \item Zoran M.Todorovic for the tabbed view, also found at CodeProject.
  \item Daniel Veillard for the excellent, high performance and small memory footprint
    XML (libxml) parser and XSLT (libxslt)\footnote{\url{http://www.xmlsoft.org/}}
    for the XSL transformation (developed for the Gnome project, ported to Win32
    by Igor Zlatkovic).
  \item Regular expressions library is Copyright \copyright{} 1993 by the Free
    Software Foundation. Iconv is also Copyright \copyright{} 2000--2002 by the
    Free Software Foundation.
  \item The YAZ library, from Index Data\footnote{\url{http://www.indexdata.dk/}},
    implements the Z39.50 protocoll. The YAZ library was recompiled by me, so the
    entire program does not depend on different VC runtime libraries.
  \item Karl Runmo for his cool HTML viewer used for the preview pane, found at
    either CodeProject or CodeGuru\footnote{\url{http://www.codeguru.com/}}, I'm
    not sure.
  \item A. Thiede for the anchoring controls on resizeable dialogs.
  \item Deepak Khajuria for the Toolbar customization.
  \item Lee Nowotny for the CListCtrl with editable subitems.
  \item Paul DiLascia for the new file dialogs on Windows~2000 and XP. If it was
    not written by him, I don't know who wrote it.
  \item Nullsoft Scriptable Install System\footnote{\url{http://nsis.sourceforge.net/}}
    is used to compile the setup program.
\end{itemize}

%%%%%%%%%%%%%%%%%%%%%%%%%%%%%%
% end of text

% You can use different bibliography styles, we use the default style `plain'.
\bibliographystyle{plain}
% Here will be the bibliography, `bibedt' is the filename of the database
% without file extension
\bibliography{bibedt}

\end{document}
